\documentclass{article}
\usepackage[polish]{babel}
\usepackage[T1]{fontenc}
\usepackage[utf8]{inputenc}
\usepackage{graphicx}
\usepackage{float}
\usepackage[bottom=1cm, right=2.5cm, left=2.5cm, top=1.5cm]{geometry}
\graphicspath{{../pliki}}



\title{%
  Cyberbezpieczeństwo - laboratoria 5 \\
  \large Analiza współczesnych algorytmów}
\author{Patryk Łuszczek 272707}
\date{\today}
\begin{document}
\maketitle
\newpage
\tableofcontents
\newpage

\section{Ocena możliwości kryptoanalizy algorytmów symetrycznych}
\subsection{Zadanie}
Do szyfrowania zostały wykozystane kolejne części klucza: \\ \textbf{5B516958F96FB744A7ECBD8DBF7A25B516958F96FB744A7ECBD8DBF7A2}

\begin{table}[H]
    \centering
    \caption{Czas potrzebny do odnalezienia pełnego klucza o różnej długości}
    \begin{tabular}{|c|c|c|c|c|}
        \hline
                      & \textbf{64} & \textbf{128} & \textbf{192} & \textbf{256} \\ \hline
        \textbf{3DES} &             & 2.5e+021     &              &              \\ \hline
        \textbf{DES}  & 1.4e+004    &              &              &              \\ \hline
        \textbf{RC4}  & 2.1e+006    & 5.6e+025     &              &              \\ \hline
        \textbf{AES}  &             & 1.8e+025     & 4.7e+044     & 9.5e+063     \\ \hline
    \end{tabular}
\end{table}

\subsection{Zadanie}

\begin{table}[H]
    \centering
    \caption{Czas potrzebny do odnalezienia klucza o długości 128bitów dla różnych algorytmów}
    \begin{tabular}{|c|c|c|c|}
        \hline
        \textbf{IDEA} & \textbf{3DES} & \textbf{TWOFISH} & \textbf{MARS} \\ \hline
        1.5e+026 lat  & 2.5e+021 lat  & 3.3e+025 lat     & 2.8e+025 lat  \\ \hline
    \end{tabular}
\end{table}
\subsection{Zadanie}
Wykorzystany został algorytm \textbf{AES (CBC)}, o kluczu \textbf{5B516958F96FB744A7ECBD8DBF7A25B5}

\begin{table}[H]
    \centering
    \caption{Czas potrzebny do odnalezienia klucza przy x nieznanych bitach}
    \begin{tabular}{|c|c|c|c|c|}
        \hline
        \textbf{4}  & \textbf{8}  & \textbf{16} & \textbf{32} & \textbf{64} \\ \hline
        natychmiast & natychmiast & ok. 0.5s    & 2h          & 1.1e+6 lat  \\ \hline
    \end{tabular}
\end{table}
\subsection{Zadanie}
Wykorzystany został algorytm \textbf{AES (CBC)}, o kluczu \textbf{5B516958F96FB744A7ECBD8DBF7A25B5}
\begin{table}[H]
    \centering
    \caption{Czas potrzebny do odnalezienia klucza przy x nieznanych bitach w różnych miejscach}
    \begin{tabular}{|c|c|c|c|c|c|}
        \hline
                          & \textbf{4}  & \textbf{8}  & \textbf{16} & \textbf{32} & \textbf{64} \\ \hline
        \textbf{początek} & natychmiast & natychmiast & ok. 0.5s    & 2h          & 1.1e+6 lat  \\ \hline
        \textbf{środek}   & natychmiast & natychmiast & ok. 0.5s    & 2h          & 1.1e+6 lat  \\ \hline
        \textbf{koniec}   & natychmiast & natychmiast & ok. 0.5s    & 2h          & 1.1e+6 lat  \\ \hline
    \end{tabular}
\end{table}
\subsection{Zadanie}
W przypadku krótkich kluczy, lub kluczy z niewielką liczbą brakujących bitów, gdzie czas łamania klucza był bardzo króki poprawny klucz został otrzymany każdorazowo.
Dla sprawdzenia działania algorytmu dla dłuższych kluczy / bardziej wybrakowanych, czas łamania byłby zbyt długo, żeby stwierdzić czy algorytm odszyfruje tekst poprawnie. Liczba szukanych bitów nie ma wpływu na jakość
odtwarzanego klucza, ponieważ i tak każdorazowo trzeba sprawdzić wszystkie możliwe kombinacje bitów. Również nie ma wpływu pozycjan nieznanych bitów z tego samego powodu - metododa brute force przeszukuje wszystkie możliwości i je sprawdza.
\subsection{Czy współczesne algorytmy blokowe możemy uznać za bezpieczne (w świetle przeprowadzonych eksperymentów)?}
Ze względu na bardzo długi czas potrzebny do złamania klucza, przy użyciu metody brute-force, algorytmy można uznać za bezpieczne.
Ważne jednak jest to, aby wykorzystać algorytmy obsługujące klucze odpowiedniu długie - dla przykładu wykorzystanie klucza 64 bitowego oznaczało, że
klucz zostanie odnaleziony w przeciągu \textbf{14 000 lat} dla algorytmu DES, co powoduje, że złamanie go jest praktycznie niemożliwe. Wykorzystanie dłuższych kluczy znacząco podnosi, czas potrzebny
do jego złamania, w związku z tym można uznać, że współczesne algorytmy są bezpieczne.

\subsection{Jaka długość klucza oferuje nam wystarczający poziom bezpieczeństwa?}
Z przeprowadzonego eksperymentu wynika, że już wykorzystanie klucza 64-bitowego zapewnia bardzo długi czas potrzebny do jego złamania metodą brute force.
Wykorzystanie dłuższych kluczy jednak znacząco podnosi czas potrzebny do złamania klucza, więc zaleca się używanie dłuższych kluczy. Dla przykładu wykorzystanie klucza 128-bitowego
oznacza, że czas potrzebny na złamanie jest rzędu od \(2.5 \times 10^{21}\) do \(1.5 \times 10^{26}\) lat, co czyni go praktycznie niemożliwym do złamania.

\subsection{Czy wielkość kryptogramu ma wpływ na możliwość jego złamania?}
Wielkość kryptogramu ma wpływ na możliwość jego złamania, ponieważ im większy jest kryptogram tym więcej informacji można z niego wyciągnąc drogą kryptoanalizu.
Dla przykładu, analiza algorytmów blokowych na większej ilości danych pozwala na dokładniejsze zidentyfikowanie cykliczności, autokorelacji, dokładniejsze n-gramy oraz bardziej wartościowe histogramy.

\subsection{Czy format i wcześniejsze przetwarzanie dokumentu (kompresja, zmiana formatu dokumentu,...) wpłwa na możliwość jego kryptoanalizy?}
Przetwarzanie dokumentu przed zaszyfrowaniem może mieć wpływ na możliwości kryptoanalizy.
Przede wszystkim kompresja dokumentu sprawia, że zawiera on mniejszą ilość informacji, co zmniejsza możliwość kryptoanalizy. Dzięki kompresji zmniejszona może zostać entropia,
histogramy, n-gramy mogą być mniej dokładne, oraz może zostać zaburzona cykliczność oraz autokorelacja.

\subsection{Ile możliwych haseł możemy sprawdzić przez rok nieustannej pracy na jednym komputerze, który sprawdza milion haseł w ciągu sekundy? Co ten wynik mówi o bezpieczeństwie współczesnych algorytmów?}
Taki komputer w ciągu roku sprawdziłby \(10^{6} \times 60 \times 60 \times 24 \times 365 =  3.1536 \times 10^{13} \) haseł. Ilość możliwych haseł dla klucza o długości 128-bitów wynosi \(2^{128} \approx 3.4028 \times 10^{38}\), więc złamanie go zajęłoby temu komputerowi
\( \frac{3.4028 \times 10^{38}}{3.1536 \times 10^{13}} \approx 1.08 \times 10^{25} \) lat. Oznacza to, że współczesne algorytmy są bardzo bezpieczne (jeśli jedyną metodą ataku jest brute-force).



\section{Ocena możliwości kryptoanalizy algorytmów asymetrycznych}
\subsection{Zadanie}
\begin{table}[H]
    \centering
    \begin{tabular}{|c|c|c|c|c|c|}
        \hline
        \textbf{Indeks} & \textbf{Rok} & \textbf{Miesiąc} & \textbf{Dzień} & \textbf{Godzina} & \textbf{Minuta} \\ \hline
        272707          & 2024         & 11               & 11             & 08               & 50              \\ \hline
    \end{tabular}
\end{table}

\[
    \begin{array}{r|l}
        272707202411110850 & 2        \\
        136353601205555425 & 5        \\
        27270720241111085  & 5        \\
        5454144048222217   & 17       \\
        320832002836601    & 2383     \\
        134633656247       & 8543     \\
        15759529           & 15759529 \\
        1
    \end{array}
\]

Znaleziono 7 czynników, narzędzie CrypTool odnalazło je w 0.022 sekundy.
\subsection{Zadanie}

\begin{table}[H]
    \centering
    \caption{Czas poszukiwania liczb pierwszych}
    \begin{tabular}{|c|c|}
        \hline
        \textbf{}{Max przedział} & \textbf{czas} \\ \hline
        \(2^{4}\)                & natychmiast   \\ \hline
        \(2^{8}\)                & natychmiast   \\ \hline
        \(2^{12}\)               & natychmiast   \\ \hline
        \(2^{16}\)               & 4s            \\ \hline
        \(2^{20}\)               & 47s           \\ \hline
        \(2^{21}\)               & 1m 25s        \\ \hline
        \(2^{22}\)               & 2m 45s        \\ \hline
    \end{tabular}
\end{table}
Czas poszukiwania rośnie wrazze wzrostem wykładnika. Jak można zauważyć czas potrzebny do odnalezienia klcuzy podwaja się,
więc można przyjąć, że dla maksymalnego przedziału \(2^{22}\) czas wynosiłby około 5 minut.

\subsection*{Ataki oparte na kracie}
Ataki oparte na kracie wykorzystują matematyczne właściwości struktur kratowych.
Z pomocą wykorzystania problemu najkrótszego lub najbliższego wektora, można ułożyć
zadanie kratowe pozwalające na złamanie pewnych aspektów algorytmu RSA, które zostaną przedstawione w dalszej części sprawozdania.
\subsection{Faktoryzacja z podpowiedzią}
Algorytm RSA wykorzystuje kika parametrów:
\begin{itemize}
    \item Liczba p oraz q - dwie liczby pierwsze
    \item Moduł N - iloczyn liczb p oraz q
    \item Eksponent e
\end{itemize}
Moduł N wraz z eksponentem e stanowią klucz publiczny (N,e), a znając wszystkie parametry można obliczyć wartość klucza prywatnego - d. Współcześnie wykorzystuje się dostatecznie duże moduły N, na przkład 1024, 2048 lub 4096 bitowe.
Wielkość N jest bezpośrednio związana z bezpieczeństwem szyfru - im większe N tym większa jest przestrzeń współczynników rozkładu, a w związku z tym trudniej jest złamać szyfrogram.
W celu rozszyfrowania szyfrogramu za pomocą ataków, N musi zostać rozłożony na czynniki pierwsze, co jest bardzo trudne i czasochłonne dla dużych liczb.
W związku z tym, że obecnie złamanym kluczem RSA, był klucz klucz 829 bitowy, zaleca się wykorzystanie kluczy przynajmniej 2048-bitowych.

W celu wykonania ataku faktoryzacji z podpowiedzią, potrzebny jest fragment tekstu jawnego oraz fragment parametru p - jednego ze składników modłułu N. Im dłuższy jest znany fragment parametru p, czyli P, tym łatwiej
jest dokonać skutecznego ataku, ponieważ liczba możliwych kombinacji maleje.
\subsection{Ataki na wiadomości stereotypowe}
Atak taki jest możliwy do zrezlizowania, tylko gdy zaszyfrowana wiadomość jest krótka, a nieznany fragment jest niewielki - czyli znana jest znaczna część tekstu jawnego.
W przypadku wykorzystania tej metody, trzeba znać pozycję oraz długość nieznanej części szyfrogramu, dlatego atak ten może być skuteczny w przypadku stereotypowych wiadomości, na przykład "Mój pin to XXXX", gdzie XXXX oznacza nieznany fragment szyfrogramu.
Więc znając szyfrogram można przeprowadzić atak na tego typu wiadomości. W przypadku bardziej złożonych wiadomości, dużego N, bardzo małego znanego fragmentu szyfrogramu, atak ten nie będzie skuteczny.
\subsection{Atak na mały tajny klucz}
Atak na mały tajny klucz, jest atakiem w wyniku którego możemy poznać współczynniki faktoryzacji modułu N. W celu przeprowadzenia ataku, musimy znać moduł N, eksponent e oraz eksponent d, który musi być stosunkowo mały w celu przeprowadzenia ataku.
Im mniejszy jest eksponent d tym łatwiejszy jest atak, ponieważ zmniejsza się wartość delta, która jest stosunkiem logarytmu z d do logarytmu z N. Program cryptool pozwala na przeprowadzenie ataku, tylko jeśli delta jest mniejsza od 0.29.

Program CrypTool umożliwia nam wygenerowanie parametrów N, e, delty oraz d, a następnie przeprowadza proces ataku. Dla domyślnych parametrów kraty, czyli m=4 (rozmiar kraty), delta może maksymalnie wynosić 0.2653. Jeśli chcemy przeprowadzić atak na większych wartościach d musimy zwiększyć rozmiar kraty.

\begin{table}[H]
    \centering
    \caption{Czas potrzebny do odnalezienia współczynników p oraz q}
    \begin{tabular}{|c|c|c|c|}
        \hline
        \textbf{Bity N} & \textbf{Delta} & \textbf{m} & \textbf{Czas} \\ \hline
        256             & 0.26           & 4          & natychmiast   \\ \hline
        512             & 0.26           & 4          & 2s            \\ \hline
        512             & 0.265          & 5          & 7s            \\ \hline
        512             & 0.27           & 6          & 25s           \\ \hline
        1024            & 0.26           & 4          & 5s            \\ \hline
        1024            & 0.27           & 6          & 1m 55s        \\ \hline
        2048            & 0.26           & 4          & 28s           \\ \hline
        2048            & 0.27           & 6          & 9m 11s        \\ \hline
    \end{tabular}
\end{table}

\textbf{Wnioski: } Im większy moduł N, tym dłużej trwa atak, ma to oczywiście związek z większą przestrzenią poszukiwania. Również widoczny jest wpływ delty na czas ataku - im mniejsza delta tym mniejszy jest współczynnik d, i łatwiej jest znaleźć współczynniki faktoryzacji.
Jeśli natomiast delta jest zbyt duża, atak może trwać bardzo długo, ponieważ dla dużych wartości d, trzeba przeprowadzić o wiele więcej operacji i zwiększyć rozmiar kraty co ma bezpośredni wpływ na szybkość wykonywania działań.

\subsection{Jaka jest minimalna długość modułu N algorytmu RSA, która gwarantuje, że jej rozkład będzie dostatecznie trudny?}
Jak już wcześniej zostało wspomniane w podpunkcie 2.3, obecnie największym złamanym modułem N był moduł 829-bitowy, co sprawia, że w przeciągu kilku lat możliwe, że zostanie złamany klucz 1024-bitowy.
W związku z tym powinno się stosować większe moduły, które będą bardziej odporne na ataki, na przykład klucz 2048-bitowy.
\subsection{Czy dla przyjętej we wcześniejszym punkcie bezpiecznej długości modułu, można przeprowadzić skuteczny atak faktoryzacji w oparciu o częściową znajomość wartości jednego z parametrów?}
Przestrzń poszukiwania klucza zależy od jego długości - im dłuższy klucz tym więcej kombinacji trzeba sprawdzić. Gdy znana jest część jednego ze współczynników, przestrzeń poszukiwań
maleje, co znacząco usprawnia poszukiwanie klucza.
\subsection{W jakich przypadkach szyfrowanie algorytmem RSA może być zagrożone przez atak realizowany w punkcie 4?}
Ataki na wiadomości stereotypowe są skuteczne, gdy zaszyfrowana wiadomość jest krótka oraz "stereotypowa", jak było już wspomniane w podpunkcie 2.4. Atak ten jest szczególnie skuteczny, dla kluczy o małej wartości, jednak
jest to wspólna cecha większości algorytmów szyfrujących.
\subsection{W jakich przypadkach szyfrowanie algorytmem RSA może być zagrożone przez atak realizowany w punkcie 5?}
W przypadku ataków na mały tajny klucz, są one skuteczne w przypadku, jak sama nazwa wskazuje, małych wartości klucza tajnego d. Wystarczająco mały klucz prywatny d znacząco zawęża
przestreń poszukiwań, nawet dla dużych modułów N.



\end{document}