\documentclass{article}
\usepackage[polish]{babel}
\usepackage[T1]{fontenc}
\usepackage[utf8]{inputenc}
\usepackage{graphicx}
\usepackage{float}
\usepackage[bottom=1.5cm, right=2.5cm, left=2.5cm, top=1.5cm]{geometry}
\graphicspath{{../pliki}}



\title{%
  Cyberbezpieczeństwo - laboratoria 8 \\
  \large Zastosowania kryptografii}
\author{Patryk Łuszczek 272707}
\date{\today}
\begin{document}
\maketitle
\newpage

\section{Jaka jest różnica między aktywnymi i pasywnymi atakami MtiM?}
Pasywny atak MtiM polega na przechwytywaniu danych komunikacji bez wpływania na ich treść. Oznacza to, że jest to atak niewidoczny z punktu widzenia ofiary, ponieważ cały proces działa "w tle". Przykładem jest zbieranie danych wrażliwych takie jak loginy czy hasła.
Z kolei atak aktywny polega na modyfikowaniu treści - podmianie komunikatu, manipulowanie zawartością czy wstrzykiwanie złośliwych treści do komunikatu. Przykładem aktywnego ataku jest np. przekierowanie ofiary do fałszywej strony, lub przeładowanie wiadomości w celu zmniejszenia wydajności.
\section{Jak zabezpieczyć swoją sieć przed atakami zatruwania ARP?}
W celu zabezpieczenia sieci przed atakami zatruwania ARP należy włączyć opcję filtrowania ARP w routerze lub ręczna konfiguracja statystycznych wpisów ARP w kluczowych urządzeniach sieci.
Innym sposobem jest wykorzystanie oprogramowania wykrywającego ARP-spoofing i zapobiegające tego typu atakom.
\section{Dlaczego ważne jest, aby używać roszerzeń DNSSEC w celub zapobiegania atakom polegającym na fałszowaniu DNS?}
Protokół DNS nie zawiera żadnych metod weryfikacji ani zabezpieczeń. W celu poprawy bezpieczeństwa należy wkorzystać rozszerzenie DNSSEC, który za pomocą
cyfrowych podpisów kryptograficznych umożliwia weryfikacje autentyczności rekordów DNS oraz umożliwia zweryfikowanie ich pochodzenia.
\section{Co to jest tryb monitorowania i jak można go używać do podsłuchiwania komunikacji siecowej?}
Tryb monitorowania to tryb pracy karty sieciowej, który umożliwia przechwytywanie wszystkich pakietów przesyłanych w sieci bez konieczności ich przetwrzania czy odsyłania.
Dzięki trybie monitorowania można podsłuchiwać komunikację sieciową oraz odczytać wiele cennych informacji (o ile nie są zaszyfrowane) takich jak pochodzenie komunikatu, docelowy odbiorca, adresy mac, hasła czy loginy.

\end{document}